
\documentclass{article}
\usepackage{gvv}
\title{\underline{\textbf{2020-7E }}}
\date{}

\begin{document}
\maketitle

\begin{enumerate}
	
\item Let $y = y(x)$ be a function of $x$ satisfying\begin{align*}y\sqrt{1 - x^2} = k - x\sqrt{1 - y^2}\end{align*} where $k$is a constant and $y\brak{\frac{1}{2}}$= - $\frac{1}{4}$.Then $\frac{dy}{dx}$ at $x = \frac{1}{2}$ is equal to:
    
\begin{enumerate}
   \item $\frac{\sqrt{5}}{2}$
   \item $- \frac{\sqrt{5}}{2}$
   \item $\frac{2}{\sqrt{5}}$
   \item $- \frac{\sqrt{5}}{4}$
\end{enumerate}
    
\item The area (in square units) of the region \begin{align*} {{(x,y)} \in \mathbb{R}^2 \mid 4x^2 \leq y \leq {8x + 12}}\end{align*} is:
    
\begin{enumerate}        
   \item $\frac{127}{3}$
   \item $\frac{125}{3}$
   \item $\frac{124}{3}$       
   \item $\frac{128}{3}$
\end{enumerate}

\item Let $\vec{a}, \vec{b}$, and $\vec{c}$ be three unit vectors such that \begin{align*}\vec{a} + \vec{b} + \vec{c} = \vec{0}.\end{align*}Let $\lambda = \vec{a} \cdot \vec{b} + \vec{b} \cdot \vec{c} + \vec{c} \cdot \vec{a}$ and $ \vec{d} = \vec{a} \times \vec{b} + \vec{b} \times \vec{c} + \vec{c} \times \vec{a}$. Then the ordered pair $(\lambda, \vec{d})$ is equal to:

\begin{enumerate}
   \item $(-\frac{3}{2}, 3 \vec{a}$
   \item $(-\frac{3}{2}, 3 \vec{c} \times \vec{b})$
   \item $(\frac{3}{2}, 3 \vec{b} \times \vec{b})$
   \item $(\frac{3}{2}, 3 \vec{a} \times \vec{c})$
\end{enumerate}
    
\item If the sum of the first 40 terms of the series:\begin{align*}3 + 4 + 8 + 9 + 13 + 14 + 18 + 19 + \dots\end{align*}is $(102)m$, then $m$ is equal to:
    
\begin{enumerate}
   \item $20$
   \item $5$
   \item $10$
   \item $25$
\end{enumerate}

\item The value of $c$ in the Lagrange's mean value theorem for the function $f(x) = x^3 - 4x^2 + 8x + 11$ when $x \in [0, 1]$ is:

\begin{enumerate}
   \item $\frac{2}{3}$
   \item $\frac{\sqrt{7} - 2}{3}$
   \item $\frac{4 - \sqrt{5}}{3}$
   \item $\frac{4 - \sqrt{7}}{3}$
\end{enumerate}

\item If $\theta_1$ and $\theta_2$ be respectively the smallest and the largest values of $\theta$ in $(0, 2\pi) - \{\pi\}$ which satisfy the equation \begin{align*}2\cot^2 \theta - \frac{5}{\sin \theta} + 4 = 0\end{align*}then the integral\begin{align*}\int_{\theta_1}^{\theta_2} \cos^2(30^\circ) \, d\theta\end{align*}is equal to:

\begin{enumerate}
   \item $\frac{2\pi}{3}$
   \item $\frac{\pi}{3} + \frac{1}{6}$
   \item $\frac{\pi}{9}$
   \item $\frac{\pi}{3}$
\end{enumerate}

\item The number of ordered pairs $(r, k)$ for which \begin{align*}6^{35} C_r = (^2 - 3)^{36} C_{r+1} \end{align*},where $k$ is an integer, is:

\begin{enumerate}
   \item 3
   \item 2
   \item 4
   \item 6
\end{enumerate}

\item Let $A = [a_{ij}]$ and $B = [b_{ij}]$ be two $3 \times 3$ real matrices such that\begin{align*}b_{ij} = 3^{(i+j-2)} a_{ij} \text{where} i,j = 1,2,3.\end{align*}If the determinant of $B$ is $81$, then the determinant of $A$ is:

\begin{enumerate}
   \item 3
   \item $\frac{1}{3}$
   \item $\frac{1}{81}$
   \item $\frac{1}{9}$
\end{enumerate}

\item Let $a_1, a_2, a_3 $ \dots be a geometric progression such that $a_1 < 0$, $a_1 + a_2 = 4$ and $a_3 + a_4 = 16$. If\begin{align*}\sum {i=1}^{9} a i = 4\lambda,\end{align*}then $\lambda$ is equal to:

\begin{enumerate}
   \item -171
   \item 171
   \item $\frac{511}{3}$
   \item -513
\end{enumerate}
7
\item Let $A, B, C,$ and $D$ be four non-empty sets. The contrapositive statement of "If $A \subseteq B$ and $B \subseteq D$, then $A \subseteq C$" is:

\begin{enumerate}
   \item "If $A \subseteq C$, then $B \subseteq C$ or $D \subseteq B$"
   \item "If $A \not\subseteq C$, then $A \not\subseteq B$ and $B \not\subseteq D$"
   \item "If $A \not\subseteq C$, then $A \subseteq B$ and $B \subseteq D$"
   \item "If $A \not\subseteq C$, then $A \not\subseteq B$ and $B \subseteq D$"
\end{enumerate}

\item If the line\begin{align*}3x + 4y = 12\sqrt{2}\end{align*}is a tangent to the ellipse \begin{align*}\frac{x^2}{a^2} + \frac{y^2}{9} = 1\end{align*}for some $a \in \mathbb{R}$, then the distance between the foci of the ellipse is:

\begin{enumerate}
   \item  $4$
   \item  $\sqrt{7}$
   \item  $2\sqrt{5}$
   \item  $2\sqrt{2}$
\end{enumerate}

\item The value of \begin{align*}4 \int_{-1}^{2} e^{-\alpha |x|} \,dx = 5\end{align*}for which $\alpha$ is:

\begin{enumerate}
   \item  $\ln\left(\frac{3}{2}\right)$
   \item  $\ln\left(\frac{4}{3}\right)$
   \item  $\log_e 2$
   \item  $\ln(\sqrt{2})$
\end{enumerate}


\item The coefficient of $x^7$ in the expression\begin{align*}(1 + x)^{10} + x (1 + x)^9 + x^2 (1 + x)^8 + \dots + x^{10} \end{align*}is:

\begin{enumerate}  
   \item  120
   \item  330
   \item  210
   \item  420
\end{enumerate}

\item Let $\alpha$ and $\beta$ be the roots of the equation $x^2 - x - 1 = 0$. If \begin{align*} P_k = \brak{\alpha}^k + \brak{\beta}^k, \quad k \geq 1\end{align*}then which one of the following statements is NOT true?        

\begin{enumerate}   
   \item $(P_1 + P_2 + P_3 + P_4 + P_5) = 26$
   \item $P_5 = 11$
   \item $P_3 = P_5 - P_4$
   \item $P_5 = P_2 - P_3$
\end{enumerate}

\item The locus of the midpoints of the perpendiculars drawn from points on the line $x = 2y$ to the line $x = y$ is:

\begin{enumerate}
   \item $2x - 3y = 0$
   \item $7x - 5y = 0$
   \item $5x - 7y = 0$
   \item $3x - 2y = 0$
\end{enumerate}

\item If $\frac{3 + i \sin \theta}{4 - i \cos \theta}$, $\theta \in [0, 2\pi]$ is a real number, then an argument of $\sin \theta + i \cos \theta$ is:

\begin{enumerate}
    \item $\tan^{-1}(3/4)$
    \item $\tan^{-1}(4/3)$
    \item $\pi - \tan^{-1}(4/3)$
    \item $\pi - \tan^{-1}(3/4)$
\end{enumerate}

\item Let $y = y(x)$ be the solution curve of the differential equation\begin{align*}(y^2 - x) \frac{dy}{dx} = 1\end{align*}satisfying $y(0) = 1$. This curve intersects the x-axis at a point whose abscissa is:

\begin{enumerate}
   \item $2+e$
   \item $2$
   \item $2 - e$
   \item $-e$
\end{enumerate}

\item  Let $f(x)$ be a polynomial of degree $5$ such that $x = \pm 1$ are its critical points. If \begin{align*}\lim_{x \to 0} \brak{ 2 + \frac{f(x)}{x^3}} = 4,\end{align*}then which one of the following is NOT true?

\begin{enumerate}   
   \item $f$ is an odd function.
   \item $x = 1$ is a point of minima and $x = -1$ is a point of maxima of $f$.
   \item $x = 1$ is a point of maxima and $x = -1$ is a point of minima of $f$.
   \item $f(1) - 4f(-1) = 4$.
\end{enumerate}

\item In a workshop, there are five machines and the probability of any one of them being out of service on a day is $\frac{1}{4}$. If the probability that at most two machines will be out of service on the same day is given by \begin{align*}\brak {\frac{3}{4}} ^3 \cdot k \end{align*}then $k$ is equal to:

\begin{enumerate}    
   \item $\frac{17}{2}$
   \item $4$
   \item $\frac{17}{8}$
   \item $\frac{17}{4}$
\end{enumerate}

\item Let the tangents drawn from the origin to the circle \begin{align*}x^2 + y^2 - 8x - 4y + 16 = 0\end{align*}touch it at the points $A$ and $B$. The value of $(AB)^2$ is equal to:

\begin{enumerate}
   \item $\frac{52}{5}$
   \item $\frac{32}{5}$
   \item $\frac{56}{5}$
   \item $\frac{64}{5}$
\end{enumerate}

\item If the system of linear equations,

\begin{align*}
    x + y + z &= 6 \\
    x + 2y + 3z &= 10 \\
    3x + 2y + \lambda z &= \mu
\end{align*}has more than two solutions, then the value of $\mu -\lambda^2$ is \rule{1cm}{0.1pt}

\item Let the function $f$ can be defined on $\brak{ -\frac{1}{3}, \frac{1}{3} }$ by\[f(x) =\begin{cases}\frac{1}{x} \ln\brak {\frac{1 + 3x}{1 - 2x} }& \text{if } x \neq 0, \\k & \text{if } x = 0.\end{cases}\]If $f$ is continuous on this interval, then $k$ is equal to \rule{1cm}{0.1pt}

\item If the mean and variance of eight numbers $3, 7, 9, 12, 13, 20, x, y$ are 10 and 25 respectively, then the value of $x  y$ is equal to \rule{1cm}{0.1pt}

\item If the foot of the perpendicular drawn from the point $(1, 0, 3)$ on a line passing through $(7,1)$ is $\brak{\frac{5}{3}, \frac{7}{3}, \frac{17}{3}}$, then the value of $a$ is equal to \rule{1cm}{0.1pt}

\item Let $X = \{ n \in \mathbb{N} \mid 1 \leq n \leq 50 \}$. 
If $A = \{ n \in X \mid n \text{ is a multiple of } 2 \}$ and  $B = \{ n \in X \mid n \text{ is a multiple of } 7 \}$,  
then the number of elements in the smallest subset of $X$ containing both $A$ and $B$ is \rule{1cm}{0.1pt}

\end{enumerate}

\end{document}                                               
