\documentclass{article}
\usepackage{amsmath, amssymb}
\title{\underline{\textbf{jee2020-paper1
}}}
\date{}

\begin{document}
\maketitle
\begin{enumerate}

\item Let $\vec{k} = \hat{i} - 2\hat{j} + \hat{k}$ and $\mathfrak{b} = \hat{i} - \hat{j} + \hat{k}$ be two vectors. If $\vec{c}$ is a vector such that $\vec{b} \cdot \vec{c} = \vec{b} \cdot \vec{a}$ and $a_0 \neq 0$, then $c_b$ is equal to:

	\begin{enumerate}
    \item $\frac{1}{2}$
    \item $ - 1$
    \item $-\frac{1}{2}$
    \item $ - \frac{3}{2}$
\end{enumerate}

\item The area (in square units) of the region $\{(x, y) \in \mathbb{R}^2 \mid x^2 \leq y \leq 3 - 2x\}$ is:

\begin{enumerate}
    \item $\frac{29}{3}$
    \item $ \frac{31}{3}$
    \item $\frac{34}{3}$
    \item $\frac{32}{3}$
\end{enumerate}

\item The length of the perpendicular from the origin to the normal of the curve $x^2 + 2xy - 3y^2 = 0$ at the point $(2,2)$ is:

\begin{enumerate}
    \item $4\sqrt{2}$
    \item $2\sqrt{2}$
    \item $2$
    \item $\sqrt{2}$
\end{enumerate}

\item If $I = \int_{1}^{2} \frac{dx}{\sqrt{2x^3 - 9x^2 + 12x + 4}}$, then:

\begin{enumerate}
    \item $\frac{1}{9} < I^2 < \frac{1}{8}$
    \item $\frac{1}{16} < I^2 < \frac{1}{9}$
    \item $\frac{1}{6} < I^2 < \frac{1}{2}$
    \item $\frac{1}{8} < I^2 < \frac{1}{4}$
\end{enumerate}

\item If a line $y = mx + c$ is a tangent to the circle $(x - 3)^2 + y^2 = 1$ and it is perpendicular to a line $L$, where $L$ is the tangent to the circle at the point $( \frac{1}{\sqrt{2}} ,\frac {1}{\sqrt{2}}
)$ ,then

\begin{enumerate}
    \item $c^2 - 6c + 7 = 0$
    \item $c^2 + 6c + 7 = 0$
    \item $c^2 + 7c + 6 = 0$
    \item $c^2 - 7c + 6 = 0$
\end{enumerate}

\item Let $S$ be the set of all functions $f: [0, 1] \to \mathbb{R}$ which are continuous on $[0, 1]$ and differentiable on $(0, 1)$. Then for every $f \in S$, there exists a $c \in (0, 1)$, depending on $f$, such that:

\begin{enumerate}
    \item $|f(c)-f(1)| <(1 - c)|f'(c)|$
    \item $f(c) - f(1) = f'(c)$
    \item $|f(c) + f(1)| > (1 + c)|f'(c)|$
    \item $\frac{f(1) - f(c)}{1 - c} = f'(c)$
\end{enumerate}

\item Which of the following statements is a tautology?

\begin{enumerate}
    \item $\neg ( p \vee \neg q) \rightarrow (p \vee q)$
    \item $\neg ( p \wedge \neg q) \rightarrow (p \vee q)$
    \item $\neg (p \vee \neg q) \rightarrow (p \wedge q)$
    \item $(p \vee \neg q) \rightarrow (p \wedge q)$
\end{enumerate}

\item If the $10^{\text{th}}$ term of an arithmetic progression is $\frac{1}{20}$ and its $20^{\text{th}}$ term is $\frac{1}{10}$, then the sum of its first 200 terms is:

\begin{enumerate}
    \item $50 \frac{1}{4}$
    \item $100 \frac{1}{2}$
    \item $50$
    \item $100$
\end{enumerate}

\item Let $f: (1,3) \to \mathbb{R}$ be a function defined by 
\[
f(x) = \frac{x - \lfloor x \rfloor}{1 + x^2}
\]
where $\lfloor x \rfloor$ denotes the greatest integer function. Then the range of $f$ is:

\begin{enumerate}
	\item $(\frac{3}{5}, \frac{4}{5})$
	\item $(\frac{2}{5},\frac{3}{5}) \cup (\frac{3}{4},\frac{4}{5})$
	\item $(\frac{2}{5},\frac{4}{5})$
	\item $(\frac{2}{5},\frac{1}{2}) \cup (\frac{3}{5},\frac{4}{5})$

\end{enumerate}

\item The system of linear equations:
\[\lambda x + 2y + 2z = 5\]
\[bx + 3y + 5z = 8\]
\[4x + ky + 6z = 10\]
has:

\begin{enumerate}
    \item Infinitely many solutions when $\lambda = 2$
    \item A unique solution when $\lambda = -8$
    \item No solution when $\lambda = 8$
    \item No solution when $\lambda = 2$
\end{enumerate}

\item If $\alpha$ and $\beta$ be the coefficients of $x$ and $x^2$ respectively in the expansion of 
	$( x + \sqrt{x^2 - 1})^6 + ( x - \sqrt{x^2 - 1})^6 $
then given

\begin{enumerate}
    \item $\alpha + \beta = 60$
    \item $\alpha + \beta = -30$
    \item $\alpha - \beta = -132$
    \item $\alpha - \beta = 60$
\end{enumerate}

\item Evaluate the limit:
\[
\lim_{x \to 0} \frac{\int_0^x t \sin(10t) \,dt}{x}
\]


\begin{enumerate}
    \item $0$
    \item $-\frac{1}{5}$
    \item $-\frac{1}{10}$
    \item $\frac{1}{10}$
\end{enumerate}

\item If 
\[
A = \begin{bmatrix} 2 & 2 \\ 9 & 4 \end{bmatrix}, \quad I = r \begin{bmatrix} 1 & 0 \\ 0 & 1 \end{bmatrix}
\]
then $10A$ is equal to:


\begin{enumerate}
    \item $41 - A$
    \item $A - 61$
    \item $61 - A$
    \item $A - 41$
\end{enumerate}

\item The mean and variance of 20 observations are found to be 10 and 4, respectively. On rechecking, it was found that an observation 9 was incorrect and the correct observation was 11. Then the correct variance is:

Options:
\begin{enumerate}
    \item $3.99$
    \item $3.98$
    \item $4.02$
    \item $4.01$
\end{enumerate}

\item If a hyperbola passes through the point $P(10,16)$ and has vertices at $(\pm6,0)$, then the equation of the normal to it at $P$ is:

Options:
\begin{enumerate}
    \item $x + 2y = 42$
    \item $3x + 4y = 94$
    \item $2x + 5y =100$
    \item $x + 3y = 58$
\end{enumerate}

\item Let $A$ and $B$ be two events such that the probability that exactly one of them occurs is  $\frac{2}{5}$ and the probability that $A$ or $B$ occurs is $\frac{1}{2}$, then the probability of both of them occurring together is:  

\begin{enumerate}
    \item $0.02$
    \item $0.01$
    \item $0.20$
    \item $0.10$
\end{enumerate}


\item The mirror image of the point $(1, 2, 3)$ in the plane $(-\frac{7}{3}, -\frac{4}{3}, -\frac{1}{3})$ is:

Which of the following points lies on this plane?

\begin{enumerate}
    \item $(-1,-1,-1)$
    \item $(-1,-1,1)$
    \item $(1,1,1)$
    \item $(1,-1,1)$
\end{enumerate}

\item Let $S$ be the set of all real roots of the equation:
\[
3x(3x-1)+2=|3x-1|+|3x-2|.
\]
Then $S$:

\begin{enumerate}
    \item is an empty set.
    \item contains at least four elements.
    \item contains exactly two elements.
    \item is a singleton.
\end{enumerate}

\item Let $\alpha = \frac{-1 + i\sqrt{3}}{2}$ if $
a = (1 + \alpha) \sum_{k=0}^{100} \alpha^{2k} \cdot a_n \cdot db = \sum_{k=0}^{100} \alpha^{3k}$
and let $a$ and $b$ be the roots of the quadratic equation. Then:

\begin{enumerate}
    \item $x^2 - 102x + 101 = 0$
    \item $x^2 + 101x + 100 = 0$
    \item $x^2 - 101x + 100 = 0$
    \item $x^2 + 102x + 101 = 0$
\end{enumerate}

\item The differential equation of the family of curves, 
\[
x^2 = 4b(y + b),
\]
where $b \in \mathbb{R}$, is:

\begin{enumerate}
    \item $x (y')^2 - x + 2y y'$
    \item $xy'' = y'$
    \item $x (y')^2 = 2y y' - x$
    \item $x (y')^2 = x - 2y$
\end{enumerate}

\item Solve for $\alpha$:
\[
H \cdot \frac{\sqrt{2} \sin \alpha}{\sqrt{1 + \cos 2\alpha}} = \frac{1}{7}
\]
and 
\[
\sqrt{\frac{1 - \cos 2\beta}{2}} = \frac{1}{\sqrt{10}}
\]
where $\alpha \in (0, \frac{\pi}{2})$. Then $\tan(\alpha + 2\beta)$ is equal to \underline{\hspace{2cm}}.

\item Let $f(x)$ be a polynomial of degree $3$ such that $f(-1) = 10$, $f(1) = -6$. The function $f(x)$ has a critical point at $x = -1$, and $f''(x)$ has a critical point at $x = 1$. Then $f(x)$ has a local minimum at $x = $ \underline{\hspace{2cm}}.

\item Let a line $y = mx$ ($m > 0$) intersect the parabola $y^2 = 3x$ at a point $P$, other than the origin. Let the tangent to it at $P$ meet the $x$-axis at the point $Q$. If $\text{area}(\triangle OPQ) = 4$ square units, then $m$ is equal to \underline{\hspace{2cm}}.

\item Evaluate the sum:
\[
\sum_{n=1}^{7} \frac{n(n + 1)(2n + 1)}{4}
\]
which is equal to \underline{\hspace{2cm}}.

\item The number of 4-letter words (with or without meaning) that can be formed from the eleven letters of the word $'\textbf{EXAMINATION}'$ is \underline{\hspace{2cm}}

\end{enumerate}

\end{document}



